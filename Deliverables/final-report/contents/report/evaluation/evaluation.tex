\chapter{Evaluation}
In this chapter we will evaluate the state of the product compared to what we had in mind at the start of the project as well as evaluate what went wrong along the way.

\section{Evaluation of the product}
The product is, in it's entirety as it is today, not exactly as we envisioned at the start of the project.

\subsection{Product Plan evaluation} We have implemented all our \emph{must haves} as listed in our Product Plan, though we haven't (fully) implemented some \emph{should haves} and \emph{could haves}, most notably the following:

\emph{The agent should be able to make decisions which have a negative effect on its goals, if it has a great benefit that can later be utilized.} This \emph{should have} may have been too ambitious and, in retrospect, should have been listed as a \emph{could have}.

Another one that was not implemented is \emph{The agent should be able to ask other parties to perform actions which are beneficial to the goals of our agent}. This was not implemented because this level of communication and interaction with other agents is not implemented in both the other bots and the connector at this time. It turned out to be very difficult to achieve, because it requires carefully devised protocol messages or natural language processing.

\subsection{Business Plan evaluation} Our team started on the project by creating a general plan of what we want our agent should be capable of, these objectives were written down in our business plan. The business plan mainly describes strategies and possible actions, the way we expected that it would satisfy our client. During the project, we found out that this business plan was very optimistic concerning the state of the connector. Almost all our objectives written in there need a lot of spatial knowledge, something that the connector is not able to provide yet. Some of our requirements were: \emph{We prefer not to build houses near student houses}, \emph{We want to make sure that our residents have parking lots near their houses}, \emph{We want to build houses on ground that is not too close to roads that create a lot of noise disturbance}, \emph{We want to build houses near enough green so the residents will have a nice environment}. All of which require a lot of spatial information and calculations, which we expected the environment would be able to provide us. After the realisation that the environment was in a very poor state, these requirements were practically given up on.

We have implemented the requirements that \emph{We must obey all local restrictions.} and \emph{We want to renovate or rebuild outdated districts whenever they do not satisfy our conditions}.

\section{Evaluation of implementation} 
This section will mainly focus on the functionality of the agent. Performance will be evaluated and failures will be analysed for each module implemented for the agent. For each of these modules a brief description of the implemented functionality is provided and is followed up by an evaluation. 

\subsection{Upgrade}
The construct module uses indicators to determine the type of building that the agent requires more of. When we own buildings that can be upgraded within a zone that has low indicator values for liveability, we will use upgrades to renovate this building. We will prioritize buildings with flat roofs in zones which also need more green, due to the fact that these can also be upgraded with a green roof after renovation. 
We want to build green roofs, because this helps the municipality accomplish their target for green, in return the municipality is more likely to approve our decisions. 
Considering the limited available upgrade types in the environment we implemented most of the beneficial possibilities. However it would be nicer to have our agent derive an optimal upgrade strategy considering more factors such as: relative location, budget, facilities and specific area strategies. To prevent buildings from being demolished after upgrading they simply can never be demolished any more. It is also possible that we upgrade a building in an area that we are trying to sell, or will be sold in the future. These interactions might lead to conflicting situation later on when a different strategy is adopted. It would be an improvement if the agent could derive a much less localized strategy instead of determining this for each building, that way conflict scenarios can be avoided. We also never look at the size of a building which we upgrade, while this can be used to improve indicators more efficiently. 

\subsection{Construct and Demolish}
The construct and demolish module primarily work together. The demolish module will attempt to demolish unwanted buildings to free up land for newer buildings to be constructed. We do so according to our building indicator, which gives the agent a global target amount for all different types of buildings. The types of buildings for which the agent has too many will be demolished if they will not be upgraded. When a buildings is demolish there is available land for constructing. The agent will determine the building that we still require the most of, according to our building indicator, and it will construct this building on the specific location. 
This implementation of these modules can be considered rather basic, because only global targets have been set. This results in somewhat randomly distribution of locations for demolishing. It would be much better if these indicators were used to determine a strategy for each zone. We were not able to develop such strategies as the connector did not provide us with enough spatial properties to derive reasonable strategies from. Additionally, we never construct a building on land where we did not previously had a building demolished. We should have implemented functionality to build on any type empty land, because we could obtain empty land by buying as well. The construction module could also be extended to choose an optimal floor size with regards to the local restrictions from the municipality. 

\subsection{Buy land and Sell land}
The buy and sell land modules are used to exchange pieces of land between stakeholders in the environment. Our sell land module finds all buildings our agent owns, which are not much use for our stakeholder but might be valuable to other stakeholders. These buildings along with the land they are located on, will be attempted to be sold what we deem as the best fitting stakeholder. Once the agent attempts to sell land it will decrease the price for this location every time an offer has been turned down, it does so three times after which it will try to sell the land to all other stakeholders, again with decreasing prices. Buy land works in a similar fashion, it also utilizes the same method to increase the price of an offer to buy land, but in this case there are only two stakeholders considered, the land owner and the agent's stakeholder. We also accept offers with a beneficial price to buy or sell land from our stakeholder, but only if we were already attempting to buy or sell this land. 
Although basic functionality is delivered, there is much more functionality achievable than present in our modules. It would be very nice, when another stakeholder sends us a request to buy or sell land, to evaluate the request based on our indicators and not only look whether we already wanted to buy or sell this land, because it will not occur often that the exact same piece of land will be requested. We also do not have any logical decision making on buying land, we should have used our land indicator to set a target amount of land owned so that we can attempt to expand our properties, this way we can use environmental factors to find land to buy. Our sell land module looks only at building types that we do not generally want for our stakeholder, such as shopping and offices. It would be a great addition if this module also looks at spatial properties to sell land with bad conditions considering its surroundings. That way we could try to achieve optimal locations for private homes, in such a way that we avoid: loud roads, student noises and green shortage in our surroundings. 

