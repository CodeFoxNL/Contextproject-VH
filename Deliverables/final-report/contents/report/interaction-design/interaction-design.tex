\chapter{Interaction Design}
This chapter describes the HCI Module that was realized for the user interaction with the developed solution. This section reports the method used to evaluate our Agent, the results gathered from conducting the method and a discussion on the way the method was conducted and the method's outcome.

\section*{Method}

\paragraph{Goal}

The goal of our contextproject is to make an agent for the Tygron environment which acts as humanlike as possible, this will be tested by asking several test subjects to play a game with the bot in the environment and react on the way the bot behaves in the environment.

\paragraph{Test subject}

The test subjects will be several different people who will get a short introduction on using the tygron environment. The subjects will not have an extended background on the topic regarding urban planning.

\paragraph{Procedure}

The test will be conducted in the following way: 
\begin{enumerate}
\item The test subject is entering a room with a computer running Tygron.
\item A brief explanation about Tygron is given regarding the actions a user can use and the way the user can improve his indicators.
\item The user gets 10 minutes to use the environment and ask questions regarding its workings. 
\item A new environment is started and agent is added to the environment.
\item The user gets about 20 minutes to play with the agent in the environment.
\item A Q\&A is conducted with the Test Subject regarding the way the other stakeholder played.
\end{enumerate}

\paragraph{Metrics}

There is only one metric used: After the Q\&A we look at the way the test subject reacted on the actions of the agent and determine from these results if he felt like he was playing against a human or against a bot. 

\section*{Results}

For now only one test subject has conducted the test, he figured out pretty quickly that it was a bot because he believed him to be \textit{too fast} and his reactions on the subjects action were \textit{too predictable}. He also stated that the agent sometimes made irrational decisions regarding use of his environment and was not very proactive towards the test subject.

\section*{Conclusion}

It was very easy for our test subject to find out that the agent indeed was a bot, this was due to his fast actions and sometimes irrational thinking.

\paragraph{Suggestions for improvement}

The most important suggestion for improvement was implementing a delay for the bot which makes him slower, the bot was \textit{very fast} and \textit{ran a lot of actions} before the subject even thought of what he had to do, the reaction time on buying the land was almost instantly when he requested to buy it which \textit{felt unhuman}, this was also true for the reaction on the bot whom was trying to buy land from our subject. Another suggestion was working on his decision making as the test subject perceived the actions as \textit{random and not very thoughtful}.

